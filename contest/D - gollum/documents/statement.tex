\documentclass{oci}
\usepackage[utf8]{inputenc}
\usepackage{lipsum}

\title{Gollum vs Golomb}

\begin{document}
\begin{problemDescription}
%\begin{center}
%	\includegraphics[scale=0.3]{gollum.png}
%\end{center}
Nuestro querido amigo Gollum se encuentra en medio de un debate de acertijos contra el hábil Golomb por la disputa del anillo único.
Luego de una infructuosa ronda de acertijos clásicos, Golomb propone el siguiente desafío a Gollum:

``Considera una secuencia de enteros positivos $G(1), G(2), G(3),\dots$ con la propiedad de que para todo valor $n \geq 1$ se tiene que $G(n) \leq G(n+1)$ y además el número $n$ aparece exactamente $G(n)$ veces en la secuencia. >Cuál es el valor de $G(10^{10})$?''

% Luego de reflexionar por un momento y de emitir sonidos de naturaleza diversa, llega a la conclusi\'on de que existe una \'unica secuencia con las propiedades demandadas por Golomb.
Gollum se estremece de emoción al escuchar tan emocionante problema.
Luego de reflexionar por un momento, y de emitir sonidos de naturaleza diversa, Gollum llega a varias conclusiones sobre la secuencia.
En primer lugar se da cuenta que $G(1)$ debe ser 1. 
Si tuviera un valor distinto este tendría que ser un número mayor que 1.
Supongamos, por ejemplo, que $G(1)$ es 3.
Por la propiedad $G(n)\leq G(n+1)$, sabemos que el 1 ya no puede aparecer en la secuencia.
Pero si $G(1)$ es igual a 3, entonces el 1 debe aparecer tres veces en la secuencia, lo cuál no es posible, pues ya habíamos concluido que el 1 no podía aparecer.
Entonces la única posibilidad es que $G(1)$ sea igual a 1.
De la misma forma Gollum se da cuenta que la única posibilidad para $G(2)$ es que sea igual a 2.
Más aún Gollum se da cuenta que los primeros valores de la secuencia son los siguientes:
\begin{center}
  \begin{tabular}{c|cccccccccccccccc}
	$n$ & 1 & 2 & 3 & 4 & 5 & 6 & 7 & 8 & 9 & 10 & 11 & 12 & 13 & 14 & 15 \\
  \hline
	$G(n)$ & 1 & 2 & 2 & 3 & 3 & 4 & 4 & 4 & 5 & 5 & 5 & 6 & 6 & 6 & 6 
  \end{tabular}
\end{center}
Gollum logra calcular mentalmente algunos valores más, pero rápidamente se da cuenta de que necesitará ser más astuto si desea calcular $G(10^{10})$.
Luego de ver su frustración, Golomb se apiada de Gollum y le permite utilizar el comodín telefónico.
Gollum recuerda que participarías en la OCI y espera que puedas ayudarlo utilizando el computador.
Tu tarea consiste en hacer un programa que dado un entero $n$ determine cuál es el valor de $G(n)$.
% Es así como has sido llamado por Gollum para ayudarlo a resolver el acertijo con la ayuda de un computador.
\end{problemDescription}

\begin{inputDescription}
  La entrada consiste en una única línea con un entero positivo $n$.
\end{inputDescription}

\begin{outputDescription}
  Tu programa debe imprimir solo un entero con el valor de $G(n)$.
\end{outputDescription}

\begin{scoreDescription}
  \score{10} Se probarán varios casos donde $0 < n \leq 15$.
  \score{25} Se probarán varios casos donde $15 < n \leq 25$.
  \score{30} Se probarán varios casos donde $25 < n \leq 10^5$.
  \score{35} Se probarán varios casos donde $10^{8} < n \leq 10^{10}$.
\end{scoreDescription}
\textbf{Nota:} En la subtarea 3 el valor de $n$ puede ser muy grande y no entrar en una variable de tipo \texttt{int}. Debes asegurarte de guardar el valor en una variable de tipo \texttt{long long}.

\begin{sampleDescription}
\sampleIO{sample1}
\sampleIO{sample2}
\sampleIO{sample3}
\end{sampleDescription}

\end{document}
