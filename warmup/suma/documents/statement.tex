\documentclass{oci}
\usepackage[utf8]{inputenc}
\usepackage{xspace}
\usepackage{lipsum}

\title{La gran suma}

\newcommand{\C}{\texttt{C}\xspace}
\newcommand{\Cpp}{\texttt{C++}\xspace}
\newcommand{\Int}{\texttt{int}\xspace}
\newcommand{\Long}{\texttt{long long}\xspace}
\newcommand{\cin}{\texttt{cin}\xspace}
\newcommand{\scanf}{\texttt{scanf}\xspace}

\begin{document}
\begin{problemDescription}
  Alejandra tiene dos números y necesita sumarlos.
El problema es que los números son muy grandes y no puede ingresarlos en su calculadora.
Alejandra espera que puedas ayudarla usando el computador.

En \C y \Cpp una variable de tipo \Int es de 32 bits, esto quiere decir que puede guardar números en el rango $[-2^{31}, 2^{31}-1]$.
El número más grande en este rango es el 2.147.483.647 que es aproximadamente $2\times 10^9$.
Si se intenta guardar un número más grande, como el 10.737.418.239, se obtendrán resultados inesperados.
Afortunadamente en \C y \Cpp es posible almacenar números más grandes usando variables del tipo \Long.
Estas variables son de 64 bits, es decir, pueden guardar números en el rango $[-2^{63}, 2^{63}-1]$.
El número más grande en este rango es el 9.223.372.036.854.775.807 o aproximadamente $9\times 10^{18}$.
Este número es bastante más grande que el máximo entero almacenable en una variable de tipo \Int.

Notar además que si se quiere sumar dos variables de tipo \Int o \Long el resultado debe entrar en una variable de tipo \Int o \Long respectivamente.
Por ejemplo, el resultado de $2^{30}+2^{30}$ es $2^{31}$ (uno más que el máximo entero de tipo \Int), y por lo tanto no se puede hacer la operación con enteros de tipo \Int.
Para realizar la operación anterior habría que guardar los enteros en una variable de tipo \Long.

Para leer un entero de tipo \Long en \Cpp puede usarse el comando \cin de la misma forma que para leer un entero de tipo \Int.
\begin{verbatim}
long long n;
cin >> n;
\end{verbatim}

En \C puede usarse la función \scanf para leer un entero de tipo \Long utilizando el modificador \texttt{\%lld}.
\begin{verbatim}
long long n;
scanf("%lld", &n);
\end{verbatim}

Tu tarea consiste en leer dos enteros y calcular su suma.
  
\end{problemDescription}

\begin{inputDescription}
  La entrada consiste en una única línea correspondiente a dos enteros $A$ y $B$ separados por un espacio.
\end{inputDescription}

\begin{outputDescription}
  Debes imprimir una única linea con el resultado de la suma de $A$ más $B$.
\end{outputDescription}

\begin{scoreDescription}
  \score{40} Se probarán varios casos donde $0\leq A,B < 2^{30}-1$.
  \score{60} Se probarán varios casos donde $2^{30}\leq A,B < 2^{62}-1$.
\end{scoreDescription}

\begin{sampleDescription}
\sampleIO{sample1}
\sampleIO{sample1}
\end{sampleDescription}

\end{document}
