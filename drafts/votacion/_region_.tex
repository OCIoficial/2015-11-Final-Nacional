\message{ !name(statement.tex)}\documentclass{article}
\usepackage[utf8]{inputenc}
\usepackage{fullpage}

\title{Votación Electrónica}

\begin{document}

\message{ !name(statement.tex) !offset(-3) }

\maketitle

La Organización de Comercio Internacional (OCI) está pasando por un proceso de
cambio de directiva. Años anteriores este proceso ha sido muy largo y tedioso,
pues miembros de todo el mundo deben votar para escoger a sus representantes. La 
directiva de la OCI está formada por 5 miembros quienes con todo su esfuerzo
intentan que la organización funcione de la mejor forma posible. Cada año un
miembro de la directiva debe dejar el cargo y se abre la posibilidad para que
otra persona postule. Es decir, es necesario escoger un único representante vez.
Como la OCI es una organización muy popular cada año postula gente de todo el
mundo al cargo directivo.  

La elección del representante es democrática y el candidato electo tiene que ser
escogido por mayoría absoluta, es decir, debe sacar más de la mitad de los
votos. Si ningún candidato obtiene más de la mitad de los votos el cupo queda
vacante.

Para simplificar el proceso, este año la organización quiere implementar un
sistema de votación electrónica. Como imaginarás, al ser una organización
dedicada al comercio, en la OCI nadie sabe programar y necesitan ayuda de gente
externa para poder implementar el sistema. Con ayuda de muchos voluntarios han
logrado completar gran parte del sistema, pero falta una parte fundamental. El
sistema puede recolectar todos los votos pero aún no es capaz de determinar si
existe un candidato que haya obtenido más de la mitad de los votos.

Dada la lista de votos tu tarea es implementar un programa que determine quién
es el ganador de la elección en caso de que exista uno. Para que un candidato
sea ganador debe obtener más de la mitad de los votos, si obtiene exactamente la
mitad este no es considerado ganador.

\section*{Entrada}
La entrada consiste en dos líneas. La primera línea contiene dos entero
positivos separados por espacio, que indican respectivamente la cantidad de
votantes $V$ y la cantidad de candidatos $C$. Los candidatos son representados
con números de 1 a $C$. La siguiente línea contiene $V$ enteros entre 1 y $C$
que representan por que candidato votó cada persona. El primer entero el
candidato por el que votó la primera persona, el segundo entero el candidato por
el que votó la segunda, etc.

\section*{Salida}


\end{document}
\message{ !name(statement.tex) !offset(-54) }
