\documentclass{article}
\usepackage[utf8]{inputenc}
\usepackage{fullpage}

\title{Escalera de la muerte}

\begin{document}
\maketitle

Han desafiado a alguien, que siempre había subido y bajado usando el ascensor, a recorrer la tenebrosa escalera de la muerte, que podría ser de subida o de bajada.
Al desafiado le gustaría saber si puede lograr el desafío sin morir en el intento así que logra conseguir/crear un mapa de la escalera, y te pide ayuda pues con tus grandiosas habilidades de programación le podrás dar la valiosa información.
El mapa es un arreglo de la altura medida en pies desde el piso en cada peldaño, y las alturas siempre son un número entero.
El desafiado es muy frágil física y mentalmente, y por tanto se debe tener las siguientes consideraciones:
No es capaz de saltar así que siempre va de peldaño en peldaño.
También se sabe que no es capaz de subir a un peldaño si está a más de 2 pies de altura respecto al anterior, y que si lo intenta su cabeza explota de la frustración.
Además, si intenta bajar 5 o más pies entre dos peldaños cae incorrectamente y su cuerpo se revienta.
Se sabe que no hay problemas para llegar al primer peldaño ni para salir del último.

\section*{Entrada}
n (cantidad de peldaños, entre 0 y 10000)
h1 h2 h3 ... hn (altura de cada peldaño en pies, entre 0 y 10000)

\section*{Salida}
"Muere en el intento :(" Si muere en el intento.
"Logra llegar al final :)" Si no.

\section*{Subtareas}
La escalera no baja entre un peldaño y otro. 50 puntos.
La escalera puede subir o bajar entre un peldaño y otro. 50 puntos.


\section*{Solución Esperada}
Recorrer el array y para cada par de peldaños ver si su diferencia está en el rango [-4, -2].


 

\end{document}
