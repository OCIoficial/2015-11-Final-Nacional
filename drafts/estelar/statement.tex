\documentclass{article}
\usepackage[utf8]{inputenc}
\usepackage{fullpage}

\title{Sistemas Estelares}

\begin{document}
\maketitle

La Oficina de Comunicaciones Interestelares, formada por distinguidos científicos de todo el mundo, se encarga del estudio de las comunicaciones a gran distancia.
Ésta es globalmente reconocida por sus equipos, que permiten la comunicación efectiva entre sus sistemas en la Tierra, sondas y estaciones espaciales.

La división de señales de alta potencia de la OCI ha descubierto un método para enviar señales con un alcance de cientos de años luz.
Para probarlo, se iniciará un programa de emisión de estas ondas de largo alcance.
Sin embargo, tal programa es muy costoso, y por ello solamente es posible emitir tales señales a cinco (5) sistemas estelares de la Vía Láctea.

Entre las divisiones menos conocidas de la OCI, se encuentra la división de Búsqueda de Vida Extraterrestre (BVET).
La principal actividad de esta división consiste en el análisis de ondas de radio provenientes de los sistemas estelares de nuestra galaxia.
Esporádicamente, aparecen pulsaciones que podrían indicar la presencia de vida extraterrestre.
Cada evento de pulsaciones se registra indicando el sistema desde el que provino, además de otra información que por el momento no nos es relevante.

BVET recomienda emitir las señales a los cinco sistemas responsables de más eventos de pulsaciones durante el último año.
Sin embargo, el equipo de informática reunció colectivamente el lunes recién pasado, por lo que la OCI ha acudido a ti con la misión de que determines estos cinco sistemas.

\section*{Entrada}
La entrada consiste en varias líneas.
La primera línea contiene un único número $N$, el número de eventos de pulsaciones ocurridas durante el último año.
Las siguientes $N$ líneas corresponden cada una a un evento de pulsaciones, y contienen un único valor $i$ ($1 \le i \le S$), el sistema en el cual se detectó el evento.

\section*{Salida}
Tu respuesta debe consistir en una única línea con cinco enteros ordenados \emph{de menor a mayor}.
Éstos deben ser los sistemas a los que se deben emitir las señales del sistema experimental.
Este conjunto siempre será único, es decir, no cabe la posibilidad de dos soluciones diferentes.

\section*{Subtareas}
\begin{itemize}
%TODO
  \item $V < 100$ y $C<100$ se puede hacer algo cuadrático
  \item $V < 10^5$ y $C<10^5$ se puede contar cada número en un arreglo
  \item $V < 10^5$ y $C<10^9$ hay que ordenarlo primero o guardar numero en un map
\end{itemize}

\end{document}
