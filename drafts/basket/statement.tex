\documentclass{article}
\usepackage[utf8]{inputenc}
\usepackage{fullpage}
\usepackage{verbatim}

\title{Partido de Basket}

\begin{document}
\maketitle

En un partido de basketball hay 5 jugadores por equipo en cancha, un director técnico tiene $N$ jugadores, y dependiendo de la corriente en el partido
el necesitará poner a ciertos jugadores con ciertas habilidades, las habilidades principales de los jugadores son ataque y defensa, estas dos habilidades
se pueden representar con el numero A y D por cada jugador, y tenemos que ayudarlo a seleccionar los 5 mejores jugadores en ataque y los 5 mejores jugadores
en defensa para que juegon cuando sea necesario.

\section*{Entrada}

Primero una linea con un numero $N$, la cantidad de jugadores que hay, a continuacion vienen N lineas que consisten en dos numeros cada uno separados por espacio
$A_i$ y $D_i$, el ataque y la defensa del $i$-esimo jugador ($i$ de 0 a $N-1$ inclusive), se garantiza q no hay jugadores con el mismo nivel de ataque y defensa, en caso de
igualar en ataque se compara la defensa y viceversa

\section*{Salida}

Hay que imprimir dos lineas, la primera con los 5 mejores jugadores en ataque y la segunda con los 5 mejores jugadores en defensa

\section*{Casos de prueba}

Input:

\begin{verbatim}
10
1 10
2 9
3 8
4 7
5 6
6 5
7 4
8 3
9 2
10 1
\end{verbatim}

Output:

\begin{verbatim}
0 1 2 3 4
9 8 7 6 5
\end{verbatim}

\section*{Solucion}
Solucion: primera subtarea puede ser ordenar en $O(N\log(N))$ (lexicograficamente) e imprimir los 5 primeros, la otra version mejorada es buscar el minimo 5 veces lo cual es $O(N)$


\end{document}
